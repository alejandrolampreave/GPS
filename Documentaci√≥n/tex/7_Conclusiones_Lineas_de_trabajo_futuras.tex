\capitulo{7}{Conclusiones y Líneas de trabajo futuras}

%Todo proyecto debe incluir las conclusiones que se derivan de su desarrollo. Éstas pueden ser de diferente índole, dependiendo de %la tipología del proyecto, pero normalmente van a estar presentes un conjunto de conclusiones relacionadas con los resultados del %proyecto y un conjunto de conclusiones técnicas. 
%Además, resulta muy útil realizar un informe crítico indicando cómo se puede mejorar el proyecto, o cómo se puede continuar %trabajando en la línea del proyecto realizado. 

\section{Conclusiones}
Una vez concluido el trabajo de fin de grado, podría decir que he aprendido más de lo esperado en un primer momento, sobre todo por el tipo de trabajo escogido, tan centrado en los sistemas de posicionamiento y en los sistemas embebidos.

El resultado (al margen de la captura de datos que se da por finalizada), es una interfaz básica, sin pretensiones de ser una herramienta completamente finalizada y con muchas líneas de trabajo futuras, pero con una buena base ya realizada, que agilice su crecimiento y multiplique sus opciones en poco tiempo.

He aprendido mucho sobre los sistemas empotrados, su versatilidad y su gran utilidad a bajo costo. También he aprendido lenguajes nuevos para mi como PHP, herramientas como XAMPP, Termite, Kinetis, Mapbox, MyGeodata o LaTeX y la importancia de las API. He  reforzado mis conocimientos en C, HTML y sistemas en tiempo real y he profundizado sobre el hecho de realizar una página de forma local para luego poder subirla a un hosting.

He alcanzado un nivel de entendimiento sobre posicionamiento GPS, que me permite comprender de una forma más clara cómo funcionan todos los GPS que nos acompañan en el día a día y lo importantes que llegan a ser, en actividades tan diferentes como operaciones de rescate o bancarias.

La organización en un proyecto de esta envergadura es primordial, de forma que planificarme y utilizar una herramienta de gestión como es GitHub, ha formado parte de mi día a día estos últimos meses y, a pesar de que requiere tiempo, finalmente quedan muy bien expuestas todas las fases por las que se ha ido pasando, más si se revisan las diferentes \textit{issues} y \textit{milestones}.


\section{Líneas de trabajo futuras}
\begin{itemize}
	\item
    Permitir la subida múltiple de ficheros.
    \item
    Crear un login de acceso.
    \item
    Crear una base de datos con todos los archivos subidos.
    \item
    Poder acceder a los archivos subidos y mostrar el que se desee.
    \item
    Añadir otros idiomas a la página.
    \item
    Implementar los botones de la interfaz que carecen de funcionalidad.
    \item
    Incluir gráficos y estadísticas.
    \item
    Cambiar la API de pago por otra gratuita o un hacer un parser que convierta los ficheros.
    \item
    Dotar a la placa de conexión a Internet para poder mostrar la ruta en tiempo real.
\end{itemize}