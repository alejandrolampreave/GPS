\capitulo{2}{Objetivos del proyecto}

En este capítulo trataremos de abordar cuáles han sido los objetivos estipulados en el proyecto que denominaremos objetivos generales y también nombraremos como objetivos técnicos aquellos que vienen dados a la hora de llevar a cabo los anteriores.

\section{Objetivos generales}
\begin{itemize}
\tightlist
\item
    Almacenar las coordenadas suministradas por el GPS en la tarjeta microSD. 
\item
	Desarrollar una web que muestre la ruta almacenada.
\item
    Poder almacenar los datos en archivos de texto a modo de histórico.
\item
    Poder descargar una imagen del mapa con la ruta.
\end{itemize}

\clearpage

\section{Objetivos técnicos}
\begin{itemize}
\tightlist
\item
	Cablear los pines Tx y Rx tanto de envío como de recepción de datos del módulo GPS con los pines digitales I/O, 0 y 1 del mismo.
\item
	Desentramar la cadena GPS transmitida por el módulo GPS, adherido a la placa de desarrollo.
\item
	Enviar los datos del módulo GPS en modo soft serial a través del puerto (UART3) y por puerto serie (UART0) al ordenador.
\item
	Guardar los mensajes NMEA recibidos en la tarjeta microSD.
\item
	Incluir un sistema operativo de tiempo real (RTOS) para la gestión de los procesos.
\item
    Desarrollar una interfaz web intuitiva y clara para el usuario que permita la subida de ficheros.
\item
	Los archivos deberán ser convertidos de NMEA a GeoJSON.
\item
	Los ficheros se guardarán en el servidor acompañados de la fecha y hora.
\item
    La ventana de la interfaz debe adaptarse a diferentes tamaños y resoluciones de pantalla.
\item
    La aplicación podrá utilizarse en cualquier dispositivo con un navegador instalado.
\item
    La aplicación deberá representar las coordenadas recogidas en el menor tiempo posible.
\item
    Emplear Git como herramienta de control de versiones incluido en la plataforma GitHub.
\end{itemize}

