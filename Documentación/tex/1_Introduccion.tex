\capitulo{1}{Introducción}


El proyecto se centra en la generación de rutas a través de un sistema embebido mediante la recolección de datos que aporta el módulo GPS adherido a la placa. Este desarrollo busca crear una herramienta que recolecte y almacene las coordenadas GPS y represente en un mapa la ruta generada. De esta manera se permite una visualización completa de todos los puntos almacenados y un histórico de los datos recogidos en diferentes días.


\section{Estructura de la memoria}

\begin{itemize}
\tightlist
\item
    \textbf{Objetivos del proyecto}: plantea los objetivos que se pretenden solucionar.
\item
    \textbf{Conceptos Teóricos}: exposición de las diferentes cuestiones teóricas vistas durante el proyecto.
\item
    \textbf{Técnicas y herramientas}: listado de las diferentes técnicas y herramientas utilizadas a lo largo del proyecto.
\item
    \textbf{Aspectos relevantes del desarrollo del proyecto}: muestra los aspectos con más relevancia en la realización del proyecto.
\item
    \textbf{Trabajos relacionados}: muestra algunos trabajos con cierto grado de similitud a la generación de rutas.
\item
    \textbf{Conclusiones y líneas de trabajo futuras}: conclusiones obtenidas en la realización del proyecto y algunas posibles ideas futuras de mejora.
\end{itemize}

\section{Estructura de los anexos}

\begin{itemize}
\tightlist
\item
    \textbf{Plan de Proyecto Software}: planificación temporal y estudio de la viabilidad legal y económica.
\item
    \textbf{Especificación de requisitos}: expone los requisitos y objetivos establecidos al principio del desarrollo.
\item
    \textbf{Especificación de diseño}: recoge la información del diseño arquitectónico y el procedimental.
\item
    \textbf{Manual del programador}: incluye los aspectos más importantes para el programador: cómo instalar las herramientas, compilar, instalar y ejecutar.
\item
    \textbf{Manual de usuario}: guía de información básica para el usuario, para un uso correcto de la aplicación.
\end{itemize}
